\documentclass[a4paper,11pt]{article}

\usepackage{german}
\usepackage{a4wide}
\usepackage[utf8]{inputenc}
\usepackage[T1]{fontenc} 
\usepackage{enumerate}
\usepackage{nicefrac}
\usepackage{lastpage}
\usepackage{fancyhdr}
\usepackage[table]{xcolor}
\pagestyle{fancy} 


\fancypagestyle{plain}{%
  	\renewcommand{\headrulewidth}{0pt}%
  	\fancyhf{}%
	\renewcommand{\headrulewidth}{0.5pt}
  	\lhead{Bitcoin User Group Stuttgart}
	\rhead{Vereinssatzung}
	\cfoot{}
	\rfoot{\thepage\ von \pageref{LastPage}}
}

\lhead{Bitcoin User Group Stuttgart}
\rhead{Vereinssatzung}
\cfoot{}
\rfoot{\thepage\ von \pageref{LastPage}}

\title{\textsf{\textbf{Vereinssatzung}}}
\author{}
\date{}

\newenvironment{bottompar}{\par\vspace*{\fill}}{\clearpage}

\begin{document}
\maketitle

\begin{enumerate}[§ 1.]
\item \textsf{\textbf{Name und Sitz}}

\begin{enumerate}[1.]
\item Der am 19. November 2017 gegründete Verein führt folgenden Namen: Bitcoin User Group Stuttgart (BUGS).
\item Der Verein soll in das Vereinsregister eingetragen werden. Ab dem Zeitpunkt der Eintragung führt der Verein den Zusatz „e.V.“
\item Sitz des Vereins ist Stuttgart.
\item Geschäftsjahr des Vereins ist das Kalenderjahr.
\end{enumerate}

\item \textsf{\textbf{Zweck des Vereins}}
\begin{enumerate}[1.]
\item Der Verein verfolgt ausschließlich und unmittelbar gemeinnützige Zwecke im Sinne des Abschnitts „Steuerbegünstigte Zwecke“ der Abgabenordnung.
\item Zweck des Vereins ist die Förderung, Erziehung, Volks- und Berufsbildung, sowie Kriminalprävention.
\item Der Satzungszweck wird insbesondere im Bereich Bitcoin und Kryptowährungen durch folgende Tätigkeiten verwirklicht:
\begin{enumerate}[i)]
\item Durchführung von Vorträgen
\item Aufklärungsveranstaltungen
\item Durchführung von kompetenzsteigernden Veranstaltungen, sogenannten Workshops
\item Der Bereitstellung einer oder mehrere Informationsplattformen, die für Diskussionen und
Weiterbildungen genutzt werden können
\end{enumerate}
\end{enumerate}

\item \textsf{\textbf{Selbstlose Tätigkeit und Mittelverwendung}}

\begin{enumerate}[1.]
\item Der Verein ist selbstlos tätig; er verfolgt nicht in erster Linie eigenwirtschaftliche Zwecke.
\item Mittel der Körperschaft dürfen nur für die satzungsmäßigen Zwecke verwendet werden. Die
Mitglieder erhalten keine Zuwendungen aus Mitteln der Körperschaft.
\item Es darf keine Person durch Ausgaben, die dem Zweck der Körperschaft fremd sind, oder durch
unverhältnismäßig hohe Vergütungen begünstigt werden.
\end{enumerate}
\newpage
\item \textsf{\textbf{Erwerb der Mitgliedschaft}}

\begin{enumerate}[1.]
\item Nur natürliche Personen können Vereinsmitglied werden.
\item Die Mitgliedschaft muss schriftlich beantragt werden und über die Aufnahme entscheidet der
Vorstand. Bei Minderjährigen haben die gesetzlichen Vertreter den Aufnahmeantrag zu stellen.
\item Der Austritt aus dem Verein ist für Mitglieder jederzeit zulässig. Der Austritt ist dem Vorstand
gegenüber schriftlich zu erklären.
\item Mitglieder deren Verhalten in grober Weise gegen die Interessen des Vereins verstößt, können
vom Verein ausgeschlossen werden. Über den Ausschluss der betroffenen Mitglieder
entscheidet die Mitgliederversammlung.
\item Die Mitgliedschaft endet mit dem Tod des Mitglieds oder Erlöschen der Mitgliedschaft.
\item Das ausgetretene oder ausgeschlossene Mitglied hat keinen Anspruch gegenüber dem
Vereinsvermögen.
\item Neben der aktiven (regulären) Mitgliedschaft, im weiteren als Vollmitgliedschaft bezeichnet, ist
eine passive Mitgliedschaft möglich. Diese entspricht der regulären Mitgliedschaft mit der Ausnahme, dass passive Mitglieder kein Stimmrecht erhalten. Andere Rechte und Pflichten bleiben hiervon unberührt.
\end{enumerate}

\item \textsf{\textbf{Beiträge}}

Vollmitglieder sind dazu verpflichtet für Ihre Mitgliedschaft Beiträge zu entrichten. Höhe, Fälligkeit und Form der Vereinsbeiträge werden von der Mitgliederversammlung bestimmt.

\item \textsf{\textbf{Organe des Vereins}}

Die Organe des Vereins sind:
\begin{enumerate}[a)]
\item Die Mitgliederversammlung
\item Der Vorstand
\item Die Ausschüsse
\end{enumerate}

\item \textsf{\textbf{Mitgliederversammlung}}

\begin{enumerate}[1.]
\item Die ordentliche Mitgliederversammlung findet zweimal jährlich statt. Des Weiteren muss eine Mitgliederversammlung einberufen werden, wenn das Interesse des Vereins es erfordert oder, wenn mindestens \nicefrac{1}{5} der Mitglieder die Einberufung schriftlich unter der Angabe des Zwecks und der Gründe verlangt.
\item Mitgliederversammlungen werden vom Vorstand schriftlich oder in Textform per E-Mail unter Angabe der Tagesordnung einberufen. Die Einladefrist zu jeder Mitgliederversammlung beträgt zwei Wochen.
\item Versammlungsleiter ist der 1. Vorsitzende. Falls der 1. Vorsitzende verhindert sein sollte, ist der 2. Vorsitzende Versammlungsleiter. Sollten weder der 1. Vorsitzende, noch der 2. Vorsitzende anwesend sein, wird ein Versammlungsleiter von der Mitgliederversammlung gewählt.
\item Sollte der Schriftführer abwesend sein, wird dieser von der Mitgliederversammlung gewählt.
\item Jede Mitgliederversammlung, die ordentlich einberufen wurde, ist ohne Rücksicht auf die
Anzahl der tatsächlich erschienenen Mitglieder beschlussfähig.
\item Die Beschlüsse der Mitgliederversammlung werden mit einfacher Mehrheit der abgegebenen
gültigen Stimmen erfasst. Jede Änderung der Satzung oder des Vereinszwecks, benötigt eine
Mehrheit von \nicefrac{2}{3} der abgegebenen gültigen Stimmen.
\item Weiterhin ist über die Beschlüsse der Mitgliederversammlung ein Protokoll zu erstellen. Das
Protokoll ist vom Versammlungsleiter und dem Schriftführer zu unterschreiben.
\item Anträge können gestellt werden von:
\begin{enumerate}[i)]
\item Jedem volljährigen Mitglied
\item Dem gesetzlichen Vertreter eines nicht volljährigen Mitglieds
\item Dem Vorstand
\end{enumerate}
\item Anträge müssen eine Woche vor der Mitgliederversammlung beim Vorstand des Vereins eingegangen sein. Wenn der Antrag später eingeht, darf dieser nur wenn die Dringlichkeit mit einer \nicefrac{2}{3} Mehrheit bejaht wird, berücksichtigt werden. Dies gilt ebenfalls für Satzungsänderungen.
\end{enumerate}

\item \textsf{\textbf{Stimmrecht und Wählbarkeit}}

\begin{enumerate}[1.]
\item Jedes Vollmitglied hat genau ein Stimmrecht.
\item Die gesetzlichen Vertreter minderjähriger Mitglieder besitzen bis zur Vollendung des 18.
Lebensjahres ein Stimmrecht.
\item Passive Mitglieder haben kein Stimmrecht.
\end{enumerate}

\item \textsf{\textbf{Vorstand}}

\begin{enumerate}[1.]
\item Der Vorstand besteht aus:
\begin{enumerate}[i)]
\item Dem Vorsitzenden (1. Vorsitzender)
\item Dem stellvertretenden Vorsitzenden (2. Vorsitzender)
\item Dem Schriftführer
\end{enumerate}
\item Der Vorstand führt die Geschäfte im Sinne der Satzung und der Beschlüsse der Mitgliederversammlung.
\item Der Vorstand ist berechtigt, für bestimmte Zwecke Ausschüsse einzusetzen.
\item Der Vorstand kann verbindliche Ordnungen erlassen.
\item Der Verein wird gerichtlich und außergerichtlich durch mindestens ein Mitglied des Vorstands vertreten.
\item Der Vorstand kann Satzungsänderungen ohne Einberufung der Mitgliederversammlung
durchführen, sofern amtliche Beanstandungen existieren. Die Änderungen dürfen nur den Zweck verfolgen besagten Beanstandungen nachzukommen. Etwaige Änderungen müssen unverzüglich allen Mitgliedern mitgeteilt werden.
\item Der Vorstand muss stets aus einer ungeraden Anzahl an natürlichen Personen bestehen. Sollte dies nicht gewährleistet sein, hat der 1. Vorsitzende, bzw. bei dessen Abwesenheit der 2. Vorsitzende, bei Stimmengleichheit die Entscheidungsmacht.
\item Die Mitglieder des Vorstands werden für jeweils zwei Jahre gewählt. Sie bleiben im Amt, bis ein neuer Vorstand gewählt ist.
\item Vorstände können beliebig oft erneut gewählt werden.
\item Das Amt des Vereinsvorstands wird grundsätzlich ehrenamtlich ausgeübt.
\item Die Mitgliederversammlung kann abweichend von Absatz 10 beschließen, dass dem Vorstand
für seine Vorstandstätigkeit eine pauschalierte und angemessene Vergütung (im Rahmen des
§3 Nummer 26a EStG) gezahlt wird.
\item Der Vorstand hat Anspruch auf Erstattung notwendiger Auslagen im Rahmen einer von der
Mitgliederversammlung zu beschließenden Richtlinie über die Erstattung von Reisekosten und
Auslagen.
\item Ist ein Vorstandsmitglied andauernd an der Ausübung seines Amtes verhindert, so muss
dieses Amt unverzüglich durch eine Wahl neu zu besetzen. Die Amtsperiode des neuen Vorstandsmitglieds endet zum ursprünglichen Ende der Amtszeit des ehemaligen Vorstandsmitglieds.
\end{enumerate}

\item \textsf{\textbf{Finanzprüfer}}
\begin{enumerate}[1.]
\item Zur Kontrolle der Haushaltsprüfung bestellt die Mitgliederversammlung Finanzprüfer. Nach Durchführung der Prüfungen setzen sie den Vorstand über das Prüfungsergebnis in Kenntnis und erstatten der Mitgliederversammlung Bericht.
\item Die Finanzprüfer dürfen nicht dem Vorstand oder einem Ausschuss angehören und können ebenfalls extern eingestellt werden.
\end{enumerate}

\item \textsf{\textbf{Auflösung, Anfall des Vereinsvermögens}}
\begin{enumerate}[1.]
\item Der Verein kann mit einer \nicefrac{3}{4} Mehrheit der abgegebenen gültigen Stimmen aufgelöst werden.
\item Bei Auflösung des Vereins oder bei Wegfall steuerbegünstigter Zwecke fällt das Vermögen des
Vereins, falls es bestehende Verbindlichkeiten übersteigt, an die Stadt Stuttgart zwecks der Verwendung für gemeinnützige Zwecke im Rahmen der Förderung von Wissenschaft und Forschung, Erziehung oder Volks- und Berufsbildung.
\end{enumerate}

\item \textsf{\textbf{Inkrafttreten}}

Die Satzung ist in der vorliegenden Form am 19. November 2017 von der Mitgliederversammlung des Vereins Bitcoin User Group Stuttgart (BUGS) beschlossen worden und tritt nach Eintragung in das Vereinsregister in Kraft.

\end{enumerate}
\include{Mitglieder}
\end{document}